\documentclass{article}


\usepackage{kr}

\usepackage{graphicx}
\usepackage{amsmath}
\usepackage{natbib}
\usepackage{xcolor}
\usepackage{amssymb}
\usepackage{subcaption}
\usepackage{algorithm}
\usepackage{algpseudocode}

\usepackage{rotating}
\usepackage{tikz}


\newtheorem{proof}{Proof}
\newtheorem{lemma}{Lemma}
\usepackage[margin=1in]{geometry}

\begin{document}


\title{Sequential Cognition Processes: A Framework For Reasoning with Non-Monotonic Logics}
\author{Axel Ind \\ axeltind@gmail.com \\ University of Freiburg}

\maketitle

\section*{Abstract}
Approaches to cognitive modelling with non-monotonic logics have thus far been largely \textit{ad hoc} and poorly standardised, making inter-model comparisons difficult. As an attempt to systematically represent non-monotonic logics in a framework that standardises cognitive modelling under these logics without sacrificing their expressiveness, we introduce the Sequential Cognition Process (SCP). Under the assumption that human reasoning can be represented as a sequence of distinct cognitive operations on an initial knowledge base SCPs provide a consistent framework for generating and evaluating models of human cognition. Using an adapted interpretation of the Weak Completion Semantics (WCS), SCPs are able to accurately model several classical experiments in cognitive modelling. We use the SCP framework to model both general case reasoners -- which arrive at the most frequently observed conclusions -- and poorly-studied individual case reasoners -- which do not. We illustrate the use of SCPs using the Suppression Task.
\end{document}
