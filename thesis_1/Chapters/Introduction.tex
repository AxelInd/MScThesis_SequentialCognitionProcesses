\chapter{Introduction} \label{chp:intro}
\section{Overview} \label{sec:overview}
% a statement of the problem
The human mind is complex. So complex that thousands of approaches from dozens of fields have failed to capture its complexity. The sheer size of the brain -- containing  over 5000 times as many neurons as the largest practical neural networks \citep{mocanu2018scalable} -- and our limited understanding of the fundamental learning processes it employs mean that researchers can neither completely describe nor predict human actions nor model human thought processes. Instead, much of the current state of the art in cognitive modelling relies on one of two general approaches: creating systems that structurally approximate the human brain, and creating systems which approximate a more abstract intuition of cognition. The first type of system encompasses many algorithms related to machine learning and deep learning; the second type, with which this paper is concerned, has existed in some form for far longer than humans have been thinking creatures \citep{smirnova2015crows}. Every shark stalking its prey, every tiny proto-mammal hiding from a hungry dinosaur, and every man driving to work in the morning, has applied this type of reasoning when trying to make predictions about the actions of other agents in their world. By applying case-specific reasoning to a known world state we are able to make imperfect, but quick, predictions about the mental state of other agents in our world.

Due in part to the difficulty of cloning dinosaurs to hunt participants in cognitive research experiments, and partly due to concerns about the ease with which they could answer questionnaires on the experience later, most cognitive tasks used by researchers tend to be more dull than the examples above.

Non-monotonic logics have proven able to adequately model a large number of standard cognitive reasoning tasks such as The Wason Selection Task\citep{wason1968reasoning} and Suppression Task \citep{byrne1989suppressing}. These approaches, though effective and well-founded in isolation, are often unable to integrate or be compared to data models of other tasks, even when they rely on the same underlying logic. Although the non-monotonic logics themselves are generally carefully described, procedures ranging from best practise in deciding appropriate knowledge bases to the mechanisms by which inferences should be interpreted tend to be re-imagined on a case-by-case basis.

Further, cognitive frameworks using non-monotonic logics are almost always designed to describe the most common (general) conclusion drawn by participants in the experiment. Modelling other individual reasoners or classes of reasoners who differ from the norm is often a non-trivial process. It has been shown that the Weak Completion Semantics \citep{holldobler2015weak} is able model the four most general cases of the Wason Selection Task under the assumption that reasoners who differ from the general case (\textit{deviant reasoners}) follow a sequence of mental processes that is still highly similar to that of the general reasoner \citep{breu2019weak}.

This thesis introduces the Sequential Cognition Process (SCP) which generalises the assumption of sequential cognitive operations, each of which uses a collection of epistemic states as input and produces a collection of epistemic states as output. 


\section{A Review of Terminology} \label{sec:terminology}
% a review of terms
\section{Current State of the Art} \label{sec:soa}
% review of literature
The current state of the art in cognitive modelling using non-monotonic logics has began to show a shift towards a computational approach to problem solving @TODOrefdiets. @TODOrefstelling argued, in a general sense, that modelling human reasoning should be done towards an appropriate representation of the underlying cognitive processes, and secondarily, it should be evaluated empirically with respect to this representation. This philosophy is in stark contrast to that of many other areas of artificial intelligence research, in which the underlying structure of the agent does not need to mimic existing cognitive processes, and instead the primary evaluation criteria is generally the effectiveness of the agent at making predictions on unseen data sets.

This thesis makes extensive reference to the work of @TODOauthornames and their contributions to the emerging field of cognitive modelling. 

The author of this thesis has previously shown the suitability of the Weak Completion Semantics for modelling individual reasoners in the Wason Selection Task \citep{breu2019weak} and the SCP framework that this thesis introduces is an extension of the intuition of discrete, sequential processes underlying cognition. The question of how to model individual reasoners is a pernicious problem in cognitive modelling and work by @TODOauthor, as well modelling tools like CCOBRA @TODOref have been able to use traditional modelling techniques, with minor variations, to replicate more than simply the most common reasoner responses.



\section{Contributions of the SCP Framework} \label{sec:contributions}
Using SCPs and a set of well-founded cognitive operations it is possible to apply traditional search techniques to problems in cognitive modelling with non-monotonic logics that have previously required expert-made models. The SCPs introduce a number of desirable properties: they introduce a partially standardised (though extensible) set of allowable cognitive operations, they standardise the structure of what constitutes an epistemic state, they are easily modified to accommodate deviant reasoners when a well-founded general model already exists, and their sequential structure makes them well-suited to scoring algorithms that allow intra- and inter-experimental modelling and comparisons.

Logics which are able to interpret conditionals of the form ``If $\phi$, then $\psi$" are varied in both their implementation and underlying principles. Work by @TODOrefslides has shown evidence that not all information is derived, stored, or retrieved symmetrically and thus, it is unlikely that a single logical system will be able fully model tasks which utilise information whose interpretation is dependant on multiple structurally and interpretationally diverse sources of information. The SCP framework enables researchers to use robust search techniques which make no assumptions about conditional interpretation at formulation time, and rather, branches to incorporate multiple possible epistemic knowledge structures at run-time. This technique, enables multiple non-monotonic logics to be used across a single cognition task.

\section{Thesis Layout} \label{sec:layout}
% description of remaining chapters
This thesis begins with a discussion of the mathematical preliminaries related to logic programming and non-monotonic logics - particularly the WCS and Reiter's default logic (Chapter~\ref{chp:prelim}). Chapter~\ref{chp:experiments} introduces several classical experiments in the field of cognitive modelling and discusses an existing non-monotonic interpretation of each. Chapter~\ref{chp:scp} introduces the SCP framework and the concept of the SCP Tasks, SCPs, and Realised SCPs. Chapter~\ref{chp:toolbox} explored the concept of the cognitive toolbox which contains well-founded tools to allow the SCP framework to model a wide variety of cognitive tasks. Chapter~\ref{chp:mimick} discussing the implementation of specific non-monotonic logics in the SCP framework. Chapter~\ref{chp:model} shows how SCPs can be used to model the classical experiments introduced in the Chapter~\ref{chp:experiments} in terms of both individual and generalised reasoners. Chapter~\ref{chp:comparing} discusses how SCPs might be compared to one another to determine which is most plausible or to identify common motifs that are present across experiments. Chapter~\ref{chp:program} briefly discusses the implementation of the SCP framework in Python~3 and the core philosophies of its design. Chapter~\ref{chp:nscp} introduces the concept of the non-Sequential Cognitive Process and its usefulness as an extension to the standard SCP framework. Finally, Chapter~\ref{chp:conclusion} summarises the contributions of this thesis and discusses potential extensions to the SCP framework.