\chapter*{Appendix} \label{chp:appendix}
\Section{Proofs}
\begin{proof} \label{proof:insertionSearch}
Given a set of cognitive operations $M$, an external activation function $f()$, and an initially valid SCP $f(\pi)$, $\pi=\{s_i \longmapsto A_0 \longmapsto ... \longmapsto A_n\}$, $A_i \in M$, inserting multiple operations into $\pi$ to create cognitive sequence $\pi'$ may result in a valid SCP $f(\pi')$, when inserting only one operation would result in an invalid SCP.

\begin{itemize}
\item Assume hybrid validity (Section~@TODOsection).
\item Let $f(\pi)=\{True if V is defined\}$, $\gamma = True$.
\item Let $A_x$ be a cognitive operation which transforms an input state point of structure $\{KB, V\}$, where $KB$ contains a set of logical formula of the form $head \leftarrow body$ and every rule in $V$ is of the form $variable \leftarrow value$, where $variable$ is an atom and  $value=\top$ into an output state point of structure $\{KB'\}$, $KB'=(KB \cup V)$.
\item Let $A_y$ be cognitive operation which transforms an input state point of structure $KB'$ into an output state point of structure $\{KB, V\}$ where $V$ contains every rule of the form $head \leftarrow body$, where $head$ is an atom and $body=\top$ and $KB= KB' \setminus V$.
\end{itemize}
\item if $f(\pi)$, $pi=(s_i)$ is a valid SCP, then $f(\pi')$, $\pi'=(s_i\longmapsto A_x)$ is an invalid SCP and $f(\pi'')$, $\pi''=(\pi'\longmapsto A_y)$ is valid.
\end{proof}

\begin{proof} \label{proof:infiniteSCPLength}
If there exists an SCP which meets goal condition $\gamma$ of length $n$, then there exists an SCP that meets goal condition $\gamma$ of length $n+k$ where $k$ is any natural number.
\begin{itemize}

\item Proof~\ref{proof:infiniteSCPs} states that if $f(x \longmapsto A)\models \gamma$ then $f(x \longmapsto T_i \longmapsto V_i)\models \gamma$ for all $(T_i \in T, P_i \in P) \in W$.
\item Let $X$ be a partial SCP of the form $X=T_0 \longmapsto V_0 \longmapsto ... \longmapsto T_N \longmapsto V_{N}$ where $(V_i, T_i) \in W$
\item For odd lengths:
\begin{itemize}
\item  It follows that if $f(x \longmapsto A)\models \gamma$ then $f(x \longmapsto A \longmapsto \textrm{exp}(X))\models \gamma$, where $\textrm{exp}$ denotes the substitution of $X$ for its constituent parts.
\item Because $X$ always contains an even number of cognitive operations, then the length of the resulting SCP must be $n' = 1+v$ where $v$ is any even number.
\end{itemize}
\item For even lengths:
\begin{itemize}
\item Notice that $f(x \longmapsto A \longmapsto \textrm{exp}(X) \longmapsto X) \models \gamma$.
\item A appending $\longmapsto X$ to adds one to the length of any SCP.
\item Because we have shown that a suitable SCP of any odd length exists $n'>=n$, it follows that a suitable SCP of any even length $n'>n$ exists.
\end{itemize}
\end{itemize}

\end{proof}

\begin{lemma} \label{lem:uniredundant}
Given a cognitive operation sequence $A \in \Omega^*$, and external function $f()$, $A$ is redundant redundant iff one of the following holds:
\begin{itemize}
\item $(x \longmapsto B \longmapsto C) \models x'$ and $(x \longmapsto A) \models x'$ for every viable epistemic state $x$, $B \in \Omega^*$, $C \in \Omega^*$. 
\item $x \longmapsto A \models x$ for every viable epistemic state $x$.
\item $f(x \longmapsto B)=c$ for all $B \in \Omega^*$, where $c$ is a constant external decision.
\end{itemize}
\end{lemma}

\begin{lemma} \label{lem:taskredundant}
Given a limited set of cognitive operations $M=\{A_0, ..., A_n\}, A_x \in \Omega^*$, and external function $f()$, $A \in M^*$ is task redundant iff one of the following holds:
\begin{itemize}
\item $(x \longmapsto B \longmapsto C) \models x'$ and $(x \longmapsto A) \models x'$ for every viable epistemic state $x$, $B \in M^*$, $C \in M^*$. 
\item $x \longmapsto A \models x$ for every viable epistemic state $x$.
\item $f(x \longmapsto B)=c$ for all $B \in M^*$, where $c$ is a constant external decision.
\item There exists no epistemic state $x$ and sequences $B, C \in M^*$ such that $f(x \longmapsto B \longmapsto A \longmapsto C)$ is a valid SCP.
\end{itemize}
\end{lemma}

\begin{proof} \label{proof:aggregateExpressiveness}
Given a cognitive task $\Pi=(x,f(), M, \gamma)$ in which $M=\{A_0,...,A_n\}$ and a second cognitive planing task $\Pi'=(x,f(),M')$ in which $M'= M \smallsetminus \{A_k,...,A_m\} \cup A'$, where $A'=(A_k \longmapsto... \longmapsto A_m)$ is a cognitive sequence containing some ordering of the other operations in $(\Pi \smallsetminus \Pi')$, $\Pi'$ is as expressive or less expressive than $\Pi$.

\item No more expressive:
\begin{itemize}
\item If $\pi' = (x\longmapsto A_p \longmapsto ... \longmapsto A' \longmapsto ... \longmapsto A_q)$ and $f(\pi)$ is a solution to $\Pi'$.
\item Let $\pi'' = (x\longmapsto A_p \longmapsto ... \longmapsto [A'] \longmapsto ... \longmapsto A_q)$.
\item Then $f(\pi'')$ is a solution to $\Pi'$ (Lemma~\ref{lemma:substitutionValid}).
\item $f(\pi'')$ is a valid solution to $\Pi$ because $\pi''$ uses only cognitive operations which occur in $\Pi$, $x$ is the initial epistemic state, and $f(\pi'') \models \gamma$. 
\end{itemize}

\item Less Expressive: Proof by counterexample
\begin{itemize}
\item If we assume that $\Pi'$ is strictly as expressive or more expressive than $\Pi$, then there exist no cases in which $\Pi'$ is less expressive.
\item Let $\gamma = \emptyset$, and $f(x)=True$ for all inputs (i.e. is trivially satisfied).
\item Let $x=\{\}$, $M=\{A_0,A_1\}$, $M'=\{A'\}$, $A'=\{A_0\longmapsto A_1\}$.
\item Then solutions of $\Pi$ are $f(x)$, $f(x \longmapsto A_0)$, $f(x \longmapsto A_1)$, $f(x \longmapsto A_0\longmapsto A_1)$ and $f(x \longmapsto A_1\longmapsto A_0)$.
\item Solutions of $\Pi'$ are $f(x), f(x \longmapsto A')$.
\item Thus, $\Pi'$ has fewer solutions that $\Pi$. A contradiction.
\end{itemize}
\end{proof}

\begin{proof} \label{proof:aggregateValid}
Given an SCP $\pi= f(x \longmapsto A_0 \longmapsto ... \longmapsto A_n)$, where $f(x)$ is an external evaluation function, $x$ is a state point, and $A_i \in \Omega$, any subsequence $A'=(A_k \longmapsto ... \longmapsto A_{k+l})$, $(k+l)<n$ of the cognitive operations in $\pi$ is a valid cognitive operation.
\begin{itemize}
\item Given that $\pi$ is a valid SCP, $x \longmapsto A_0$ must, by the definitions discussed in Section~@TODOref result in a valid state point.
\item It follows that $A'$ takes a valid state point as input, because $A_k$ took a valid state point as input.
\item It follows that $A'$ produces a valid state point as output, because $A_{k+1}$ produced a valid state point as output.
\item Therefore, $A'$ is a valid cognitive operation, by the definition in Section~@TODOref.
\end{itemize}
\end{proof}

\begin{lemma} \label{lemma:substitutionValid}
Let $\pi=(x\longmapsto A_0, ..., A', ..., longmapsto A_n)$ be a sequence of cognitive operations drawn from some cognitive task $\Pi=(x, \gamma, f(x), M)$ . And let $\pi'=(x\longmapsto [A_0], ..., A', ..., longmapsto A_n)$ where $[A]$ is the substitution of the aggregate operation $A'$ for $(A'[0]\longmapsto ... \longmapsto A'[t])$ . Then $f(\pi)=f(\pi')$.
\end{lemma}

\begin{proof} \label{proof:infiniteSCPs}
There are an infinite number of possible, valid SCPs, that can meet any goal condition $\gamma$ from input $x$ provided that at least one SCP exists that can reach goal condition $\gamma$ from input $x$.
\begin{enumerate}
\item There exist infinitely many cognitive operations $T \in \Omega$ which add a new variable from the set of all possible variable names $p \in P$.
\item There exist infinitely many cognitive operations $V \in \Omega$ which remove a variable from the set of all possible variable names  $p \in P$.
\item It follows that there exist infinitely many pairs $(T_i \in T, P_i \in P) \in W$ where $T_i$ adds a variable $p \in P$ to the resulting state point, and $V_i$ removes that variable from the resulting statepoint.
\item Then $f(x \longmapsto A) = f(x \longmapsto T_i \longmapsto V_i)$
\item Thus it follows that if $f(x \longmapsto A)\models \gamma$ then $f(x \longmapsto T_i \longmapsto V_i)\models \gamma$ for all $(T_i \in T, P_i \in P) \in W$.
\end{enumerate}
\end{proof}

\Section{Output}
===========================================================
=================ABDUCIBLE SUPPRESSION=====================
===========================================================
Epistemic state:

===>el<===
S:: [(e ↔ ⊤), (l ↔ (e ∧ (¬ ab_1))), (ab_1 ↔ ⊥)]
Delta:: []
V:: [(e:True), (l:True), (ab_1:False)]
R:: {'abducibles': []}

response: She will study late in the library
------------------------------
Epistemic state:

===>el<===
S:: [(e ↔ ⊤), (l ↔ (e ∧ (¬ ab_1))), (ab_1 ↔ ⊥), (o ↔ ⊤)]
Delta:: []
V:: [(e:True), (l:True), (ab_1:False)]
R:: {'abducibles': [(o ← ⊤)]}

response: She will study late in the library
------------------------------
Epistemic state:

===>el<===
S:: [(e ↔ ⊤), (l ↔ (e ∧ (¬ ab_1))), (ab_1 ↔ ⊥), (o ↔ ⊥)]
Delta:: []
V:: [(e:True), (l:True), (ab_1:False)]
R:: {'abducibles': [(o ← ⊥)]}

response: She will study late in the library
------------------------------
Epistemic state:

===>el<===
S:: [(e ↔ ⊤), (l ↔ (e ∧ (¬ ab_1))), (ab_1 ↔ ⊥), (o ↔ (⊤ ∨ ⊥))]
Delta:: []
V:: [(e:True), (l:True), (ab_1:False)]
R:: {'abducibles': [(o ← ⊤), (o ← ⊥)]}

response: She will study late in the library
------------------------------
Epistemic state:

===>elo<===
S:: [(e ↔ ⊤), (l ↔ ((e ∧ (¬ ab_1)) ∨ (o ∧ (¬ ab_2)))), (ab_1 ↔ (¬ o)), (ab_2 ↔ (¬ e))]
Delta:: []
V:: [(e:True), (l:None), (o:None), (ab_1:None), (ab_2:False)]
R:: {'abducibles': []}

response: We are uncertain if she will study late in the library
------------------------------
Epistemic state:

===>elo<===
S:: [(e ↔ ⊤), (l ↔ ((e ∧ (¬ ab_1)) ∨ (o ∧ (¬ ab_2)))), (ab_1 ↔ (¬ o)), (ab_2 ↔ (¬ e)), (o ↔ ⊤)]
Delta:: []
V:: [(e:True), (l:True), (o:True), (ab_1:False), (ab_2:False)]
R:: {'abducibles': [(o ← ⊤)]}

response: She will study late in the library
------------------------------
Epistemic state:

===>elo<===
S:: [(e ↔ ⊤), (l ↔ ((e ∧ (¬ ab_1)) ∨ (o ∧ (¬ ab_2)))), (ab_1 ↔ (¬ o)), (ab_2 ↔ (¬ e)), (o ↔ ⊥)]
Delta:: []
V:: [(e:True), (l:False), (o:False), (ab_1:True), (ab_2:False)]
R:: {'abducibles': [(o ← ⊥)]}

response: She will not study late in the library
------------------------------
Epistemic state:

===>elo<===
S:: [(e ↔ ⊤), (l ↔ ((e ∧ (¬ ab_1)) ∨ (o ∧ (¬ ab_2)))), (ab_1 ↔ (¬ o)), (ab_2 ↔ (¬ e)), (o ↔ (⊤ ∨ ⊥))]
Delta:: []
V:: [(e:True), (l:True), (o:True), (ab_1:False), (ab_2:False)]
R:: {'abducibles': [(o ← ⊤), (o ← ⊥)]}

response: She will study late in the library
------------------------------
predictions:  {'el': ['She will study late in the library', 'She will study late in the library', 'She will study late in the library', 'She will study late in the library'], 'elo': ['We are uncertain if she will study late in the library', 'She will study late in the library', 'She will not study late in the library', 'She will study late in the library']}