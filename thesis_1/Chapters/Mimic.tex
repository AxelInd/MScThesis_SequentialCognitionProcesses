\chapter{Modelling Experiments with SCPs} \label{chp:model}
\section{Overview}
As with any cognitive framework, the most important metric for judging the success of SCPs comes from testing how well SCPs can approximate empirical data across tasks. This chapter will show the suitability of SCPs with a common set of allowable cognitive operations for modelling several well-studied experiments in cognitive modelling.

In particular, we will show that the Wason Selection Task, Suppression Task and @TODO can be modelled at both the general and individual reasoner level with SCPs in a formulation that intuitive and consistent with the WCS approach already described in Section~@TODOref.
\section{Suppression Task}



To model this experiment in the SCP framework, we must first define the SCP Task $\Pi_{sup}=(s_i,M,f(),\gamma)$ which outlines the restrictions and goals of the model. The structure of $s_i$ must be robust enough to capture information regarding a set of rules, variable assignments in the possible world, and conditionals. To this end we define:

\[s_i^\text{noSup}=\{S,\Delta_\text{noSup}, V\} \]

\[s_i^\text{sup}=\{S,\Delta_\text{sup}, V\} \]

For the the only unconditional rule, that she has an essay to write, we define $S=\{(e \leftarrow \top)\}$. The set of conditional rules $\Delta_{\text{noSup}}=\{l|e\}$ captures the probable assumption that she will study late in the library if she has an essay to write, $\Delta_{\text{sup}}=\{(l|e),(l|o)\}$ captures the addition of the conditional that if the library is open she will study late in the library. And the possible world $V=\{(e,u),(l,u)\}$ sets all variable names occurring in $S$ and $\delta$ to unknown.

The second consideration $M$, denotes the set of possible cognitive operations which may occur in the resulting SCPs and realised SCPs of $\Pi_{sup}$. Following the intuition we used in Section~@TODOref we choose cognitive operations that enable application of the WCS and interpretation of conditionals. To that end we define:

\[
M=\{\texttt{addAB}, \texttt{semantic}, \texttt{wc}\}
\]

The external evaluation function should translate the SCP into a decision that can be compared with empirical data. In the Suppression Task we will treat $f()$ as the question: `Will she study late in the library?'. Under the assumption only the final state point $p_n$
determines whether she will study late, we define:

\[
f(\pi) = \begin{pmatrix} \text{she will study late in the library} & (l,\top) \in p_n[V] \\   \text{she will not study late in the library} & (l,\bot) \in p_n[V] \\ 
\text{we are unsure if she will study late in the library} & \text{otherwise} \end{pmatrix}
\]

Next, we define the empirical result we wish to emulate $\gamma$. In this case we will define $\gamma_{text{noSup}}$ to model cases where no suppression is observed, and $\gamma_{\text{sup}}$ to model cases where suppression is observed.

\[\gamma_{text{noSup}} = \text{she will study late in the library} \]
\[\gamma_{text{sup}} = \text{she will not study late in the library} \]

Armed with these definitions, it is possible to model the case where she will study late in the library $\Pi$ as follows:

\[\Pi_\text{noSup}=(s_i^{\text{noSup}},M,f(),\gamma_{\text{noSup}})\]

De Novo search on $\Pi_\text{noSup}$ returns the satisfying SCP $\mu_\text{noSup}$:

\[\mu_\text{noSup}=(\pi=(s_i^{\text{noSup}} \longmapsto \texttt{addAB} \longmapsto \texttt{wc} \longmapsto \texttt{semantic}),f())\]

Figure~@TODOref illustrates the realised SCP for $\mu_\text{noSup}$ and illustrates that $\mu_\text{noSup} \models \gamma$.

The next case to examine is the one in which we show evidence of suppression in the SCP framework $\Pi_\text{sup}$. To that end we find satisfying SCPs for the task:

\[\Pi_\text{sup}=(s_i^{\text{sup}},M,f(),\gamma_{\text{sup}})\]

De Novo search on $\Pi_\text{sup}$ returns the satisfying SCP $\mu_\text{noSup}$:

\[\mu_\text{sup}=(\pi=(s_i^{\text{sup}} \longmapsto \texttt{addAB} \longmapsto \texttt{wc} \longmapsto \texttt{semantic}),f())\]

\begin{figure}
\centering \includegraphics[scale=0.4]{Suppression_SCP}
\caption{Suppression for the two SCPs $\mu_\text{noSup}$ and $\mu_\text{sup}$}.
\label{fig:Suppression_SCP}
\end{figure}

Figure~\ref{Suppression_SCP} illustrates the realised SCP for $\mu_\text{noSup}$ and $\mu_\text{sup}$ and illustrates a suppression effect between the two.

These two examples have served to show that suppression can modelled in the SCP framework. The reader might be tempted to argue that the SCP $\mu_\text{ce1} = (\pi_\text{ce1}=(s^\text{sup}_i),f())$ also demonstrates Suppression because $f(\pi_\text{ce1})\models \gamma_\text{sup}$. However, this is not the case because Suppression occurs only when inferences drawn from a knowledge base \textit{no longer} occur. Using the same SCP without the additional conditional in the epistemic state results in $\mu_\text{ce2} = (\pi_\text{ce2}=(s^\text{noSup}_i),f())$ and $f(\pi_\text{ce2})=f(\pi_\text{ce1})$. Suppression is only demonstrated when the inference \textit{changes} when the initial state is given additional information.

To illustrate the ability of the framework to model individual reasoners, we examine those reasoners who despite knowing both that if she has an essay to write she will study late in the library and that if the library is open she will study late in the library still do not demonstrate suppression, and instead draw the classical conclusion that she will study late in the library. To this end we define a third SCP Task $\Pi_\text{deviant}$.

\[\Pi_\text{deviant}=(s_i^{\text{sup}},M,f(),\gamma_{\text{noSup}})\]

De Novo search on $\Pi_\text{sup}$ returns no satisfying SCP for this task and it is clear that some change must be made, either to the knowledge contained in the initial state, or to the set of allowable operations $M$. The initial state could be supplemented by additional background knowledge, but we cannot be certain of this background knowledge and instead will focus on finding appropriate cognitive operations to add to our model. We create $M'=M\cup \texttt{delete}_o \cup \texttt{fix}_{(\text{ab}_1\leftarrow \bot)}$. $M'$ is now able to delete the variable $o$ and to force the $V[\text{ab}_1]$ to always return $(\text{ab}_1,\bot)$. To see the explanations and justifications for these two cognitive operations, refer to Section~@TODOref, and Section~@TODOref.

Equipped with these two cognitive operations, there are now two SCPs of length $4$ which can model the SCP Task $\Pi'_\text{deviant}$.

\[\Pi'_\text{deviant}=(s_i^{\text{sup}},M',f(),\gamma_{\text{noSup}})\]

\[\mu^1_\text{deviant}=(\pi^1_\text{deviant}=(s_i^{\text{sup}} \longmapsto \texttt{delete}_o \longmapsto \texttt{addAB} \longmapsto \texttt{wc} \longmapsto \texttt{semantic}),f())\]
\[\mu^1_\text{standard}=(\pi^1_\text{standard}=(s_i^{\text{noSup}} \longmapsto \texttt{delete}_o \longmapsto \texttt{addAB} \longmapsto \texttt{wc} \longmapsto \texttt{semantic}),f())\]


\[\mu^2_\text{deviant}=(\pi^1_\text{deviant}=(s_i^{\text{sup}} \longmapsto \texttt{addAB} \longmapsto \texttt{delete}_o \longmapsto \texttt{wc} \longmapsto \texttt{semantic}),f())\]


\[\mu^3_\text{deviant}=(\pi^1_\text{deviant}=(s_i^{\text{sup}} \longmapsto \texttt{addAB} \longmapsto \texttt{wc} \longmapsto  \texttt{delete}_o \longmapsto \texttt{semantic}),f())\]

\[\mu^4_\text{deviant}=(\pi^1_\text{deviant}=(s_i^{\text{sup}} \longmapsto \texttt{fix}_{(\text{ab}_1\leftarrow \bot)} \longmapsto \texttt{addAB} \longmapsto \texttt{wc} \longmapsto \texttt{semantic}),f())\]

\[\mu^5_\text{deviant}=(\pi^1_\text{deviant}=(s_i^{\text{sup}} \longmapsto \texttt{addAB} \longmapsto \texttt{fix}_{(\text{ab}_1\leftarrow \bot)} \longmapsto \texttt{wc} \longmapsto \texttt{semantic}),f())\]



\[\mu^6_\text{deviant}=(\pi^1_\text{deviant}=(s_i^{\text{sup}} \longmapsto \texttt{addAB} \longmapsto \texttt{wc} \longmapsto \texttt{fix}_{(\text{ab}_1\leftarrow \bot)} \longmapsto \texttt{semantic}),f())\]

Figure~\ref{fig:Suppression_SCP_del1} illustrates derivation of the classical logical response to the suppression task as a modification of an SCP which did illustrate suppression for the two SCPs $\mu^1_\text{deviant}$ and $\mu^1_\text{standard}$. This evidences one of the core claims of the SCP Framework which is that it is a reasonable tool for modelling individual reasoners in a cognitive task. 

Figure~@TODOref illustrates the same phenomenon for the SCPs @TODO and @TODO using the $\texttt{fix}_{(\text{ab}_1\leftarrow \bot)}$ cognitive function.

The value of these examples lies in that they show that classical logic response to a cognitive task is not necessarily derived through the use of classical logic.

\begin{figure}
\centering \includegraphics[scale=0.4]{Suppression_SCP_del1}
\caption{Inhibition of Suppression for the two SCPs $\mu^1_\text{deviant}$ and $\mu^1_\text{standard}$}.
\label{fig:Suppression_SCP_del1}
\end{figure}



\subsection{Extending the Suppression Task with SCPs}
The previous example merely showed that SCPs are suitable for modelling the suppression task. In this example we consider one of the most powerful characteristics of SCPs, the ability to model unusual results as deviations from general reasoning. In the original Suppression Task Experiment examined in Section~\ref{sec:sup}, a significant portion of people still believed that she would study late in the library, even though the majority suppressed the inference. Several possible explanations are intuitive, the first and simplest, is the assumption that the reasoner is using classical logic and drawing the classical conclusion. However, what if that is not the case? What if they do reason in exactly the same way as the other reasoners, except for one or two small deviations?

In order to model these non-general reasoners, we consider two possible deviations that could explain the classical result of the Suppression Task: \textit{variable deletion}, and\textit{ variable fixation}. Both of these operations will be discussed in a way that may seem overly prosaic, but it is done to reinforce that we might, reasonably, expect these cognitive operations to occur in day-to-day human cognition.


\begin{figure*}
\begin{center}
\includegraphics[width=0.85\linewidth]{suppressionSCP_mod}
\end{center}

\caption{The Suppression Task in which the additional operation of deleting the variable $o$ occurs.}
\label{fig:supmod}
\end{figure*}

\begin{figure*}
\begin{center}
\includegraphics[width=\linewidth]{suppressionSCP_mod2}
\end{center}

\caption{The Suppression Task in which the additional operation of fixing the variable $ab_1$ to false occurs.}
\label{fig:supmod2}
\end{figure*}
\section{Wason Selection Task}

\[\Pi=(s_i,M,f(),\gamma\]



\[\mu=(\pi,f())\]

The choice of external evaluation function for this task (the turn function), if we are to follow the intuition from @TODO, has two requirements: it must be able to capture whether an observation $O$ can be explained by the least model of the weakly completed program, and it must ensure that variable assignments which,if incorrect could falsify the rule $2\leftarrow D$, are present in the least model. To this end, we add an observation variable $O$ to the definition of the initial epistemic state, and combine this with the variables for the set of propositional rules $S$, conditional rules $\Delta$, and possible world $V$.



\[f(\pi,O)==  I_V(O)\models \top \textrm{ and } I_V(D)\models \top \lor \bot \textrm{ and } I_V(3)\models \top \lor \bot\]



Next, we require an intuition for the generation of the initial epistemic state point $s_i$. In this case, abduction can be simulated by the SCP in one of three ways. The first is to create a unique SCP for each possible explanation for each observation as we did for the classical WCS interpretation of the task in Section~@TODOref. But the fact that SCPs may contain an initial state point, rather than just a single epistemic state, enables a more elegant solution that captures the full scope of possible observation, explanation pairings. The other two options are to start with a state point containing each abductive case, or to create cognitive operations which create these abductive cases at computation time.

Because SCPs aim to capture as much cognitive information as possible within the CTM as possible, we will prefer this third case and make use of a categorization variable $R$ to specify both the set of observations $O$, and the possible explanations $\eta$. The cognitive operation \texttt{AddExp} is created.

\texttt{AddExp} adds possible explanations $\eta$ to the set of rules $S$ in the epistemic state and is defined as follows:

%------
\[\texttt{AddExp}(\textit{input})=(\chi,e)\]
\[\chi=(\textit{input}\models (S,R))\]
\[e=\]
\[R'\subseteq R[\textrm{'explanations'}]\]
\[d := (R'[1]\cup...\cup R'[n])\]

\[S:=S\cup d\]

\[R[\epsilon]:=R[\epsilon] \cup e\]
@TODOdefine$\cup$forR
%------

Figure~@TODOref shows the state point that results from applying this cognitive operation to $s_i=(S=\{\},\Delta=\{(3|D)\},V=\{D,K,3,7\},O=\{\},R=\{'explanations':\{D,K,3,7\}\})$ for cases where $R'$ is a single abducible.

Because the intention of this task is model a known response set for the WST, we must consider some way of generating an SCP in which $f(\pi)= \gamma$. As before this necessitates an appropriate selection of cognitive functions for $M$. We will make use of the set:

\[
M=\{\texttt{addAB},\texttt{wc},\texttt{semantic}, \texttt{addExp}\}
\]

Conducting de Novo search on $\Pi$ for each of the observations at this point now returns a state point containing the following CTMs:
\[
\pi_1=(s_i \longmapsto \texttt{addAB} \texttt{addExp} \longmapsto \longmapsto \texttt{wc} \longmapsto \texttt{semantic}))
\]

\[
\pi_2=(s_i \longmapsto \texttt{addExp} \longmapsto \texttt{addAB} \longmapsto \texttt{wc} \longmapsto \texttt{semantic}))
\]


\[
\mu_D^1=(\pi_1), f(\pi, O=\{D\}))
\]
\[
\mu_D^2=(\pi_2), f(\pi, O=\{D\}))
\]

\[
\mu_K^1=(\pi_1), f(\pi, O=\{K\}))
\]
\[
\mu_K^2=(\pi_2), f(\pi, O=\{K\}))
\]

\[
\mu_3^1=(\pi_1), f(\pi, O=\{3\}))
\]
\[
\mu_3^2=(\pi_2), f(\pi, O=\{3\}))
\]

\[
\mu_7^1=(\pi_1), f(\pi, O=\{7\}))
\]
\[
\mu_7^2=(\pi_2), f(\pi, O=\{7\}))
\]

\begin{figure}
\centering{\includegraphics[width=\linewidth]{realisedSCPsWST}}
\label{fig:rSCP_WST}
\caption{Realised SCPs for the SCP interpretation of the WST $\pi_1=(s_i \longmapsto \texttt{addAB} \longmapsto \texttt{addExp}  \longmapsto \texttt{wc} \longmapsto \texttt{semantic}))$. Results mimic those observed in @TODOref's WST interpretation.}
\end{figure}

These results are not surprisingly and illustrate two important points. The first is that, for external evaluation functions that only evaluate the final epistemic state $p_n$, $f(\pi_1,O=\{o\})=f(\pi_2,O=\{o\})$ for any observation $o$. Secondly, the non-monotonic cognitive operation \texttt{addExp} means that these SCPs do not necessarily produce a single base point for $p_n$ and it will be necessary to examine the realised SCPs to determine which cards can actually be turned. Figure~\ref{fig:rSCP_WST} shows the realised SCP that can be produced for explanations of length 1 (as with @TODOcite's original WCS interpretation minimal explanations are preferred) using CTM $\pi_1$. @TODOdefineindependenceofcognitiveoperations. 

\begin{figure}
\centering{\includegraphics[width=\linewidth]{realisedSCPsWST_simple}}
\label{fig:rSCP_WST_simple}
\caption{Realised SCPs for the simplified SCP interpretation of the WST $\pi_1=(s_i \longmapsto \texttt{addAB} \texttt{addExp} \longmapsto \longmapsto \texttt{wc} \longmapsto \texttt{semantic}))$.}
\end{figure}

In order to model the three deviant cases of WST, we follow the same reasoning discussed in Section~@TODOsec and introduce two extensions to our original model: Basic Weak Completion, and Contraposition.

Basic Weak Completion is achieved by formulating a new external evaluation function $f'(\pi)=f(\pi,O=\epsilon)$. Because $I_V(\top)\models \top$, the observation is trivially satisfied and the evaluation function simply depends on whether the possible world of realised SCP assigns values which, if they were otherwise, could violate the conditional $(3|D)$. Figure~\ref{rSCP_WST_simple} illustrates $f'(r)$ for each of the realised SCPs generated in the abductive interpretation of SCPs, where the explanation $\epsilon$  is minimal.

Contraposition can be achieved by creating a new congitive operation \texttt{contra} which applies add the classical modus tolens implications to all conditionals contained in $\Delta$ of the input state.

%------
\[\texttt{contra}(\textit{input})=(\chi,e)\]
\[\chi=(\textit{input}\models (\Delta))\]
\[\Delta' := \Delta\]
\[\forall_{(\psi|\phi) \in \Delta'}\]:
\begin{itemize}
\item $S:=S \cup (\phi'\leftarrow \lnot \phi)$
\item $\Delta := \Delta\cup (\phi'| \lnot \psi)$
\item $V:=V\cup(\phi':u)$
\end{itemize}
%------



















