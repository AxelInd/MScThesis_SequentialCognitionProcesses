\chapter{Program Design} \label{chp:program}
\section{Implementing the SCP Framework}
The SCP framework has been implemented in \texttt{Python 3} as series of modules which follow strong object-oriented programming principles which support simple extensions and revisions. This implementation is used to demonstrate the practical approaches to generating and evaluating SCPs; and to model a series of cognitive tasks for which empirical results already exist. The library containing this implementation and its associated documentation is available at @TODOref.

At present, concrete implementations for epistemic state structures and cognitive operation formulations related to propositional logic, the WCS, and Reiter's default logic are provided.

\section{A Modular Programming Approach}
% talk about each 
The principle of modular programming is followed in this implementation to the greatest extent possible, thus maximising the ease of modification and upkeep for the program. Figure~@TODOref provides a simplified class diagram to explain how the major components of the system relate to one another.

\subsection{Most Important Modules}

\subsection{Program Flow}
Diagram~@TODOref demonstrates the basic programatic flow used to model the Suppression Task using a WCS SCP. Diagram~@TODOref demonstrates the programmatic flow used by the De Novo search in order to generate an SCP to map to known data. Finally, Diagram~@TODOref shows the process by which information stored in a .csv file can be read into CCOBRA before being interpreted by the SCP framework and used for modelling.

\section{External Framework Integration}

