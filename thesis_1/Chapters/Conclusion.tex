\chapter{Conclusions and Future Work} \label{chp:conclusion}
\section{Conclusion}
SCPs represent a novel and powerful framework for modelling non-monotonic logics. They have been shown capable of modelling the Suppression Task and Wason Selection under the WCS for both general and individual cases. SCPs provide a dynamic framework incorporating cognitive operations which are applicable across different logics, provided that those logics share structural features that are compatible with some subset of the known set of cognitive operations. Because the input and output epistemic states need not be structurally similar, SCPs may represent the first approach to modelling human cognition that is able to integrate multiple non-monotonic logic frameworks at run-time and at search-time. 

SCPs represent a new frontier for Cognitive Modelling in non-monotonic logics and research into their capabilities and limitations may help create more robust and mathematically consistent explanations and predictions for human behaviour across an extensive array of cognitive tasks.

\section{Future Work}
First Order Logic
Machine Learning
nSCPs
Extend the set of cognitive operations
Heuristics for SCPs
